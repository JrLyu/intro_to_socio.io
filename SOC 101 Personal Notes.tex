\documentclass[12pt,a4paper]{article}

\usepackage[utf8]{inputenc}
\usepackage{mathtools}
\usepackage{amssymb}
\usepackage{ntheorem}
\usepackage[framemethod=TikZ]{mdframed}
\usepackage{amsmath}
\usepackage[hidelinks]{hyperref}
\usepackage{cleveref}
\usepackage[most]{tcolorbox}
\usepackage{fancyhdr}
\usepackage{lastpage}
\usepackage{geometry}
\usepackage{graphicx}
\usepackage{float} 
\usepackage{subfigure} 
\usepackage{arydshln}
\usepackage{multicol}
\usepackage{url}
\usepackage{setspace}
\usepackage[T1]{fontenc}
\usepackage{mathptmx}
\usepackage{framed}
\usepackage{xcolor}
\usepackage{listings}
\usepackage{chemfig}

\geometry{a4paper, left=2cm, right=2cm, bottom=2cm, top=2cm}

\definecolor{darkblue}{HTML}{003472}
\definecolor{lightblue}{HTML}{70f3ff}
\definecolor{red}{HTML}{dc3023}
\definecolor{lightred}{HTML}{f47983}
\definecolor{green}{HTML}{057748}
\definecolor{lightgreen}{HTML}{a4e2c6}
\definecolor{gray}{HTML}{758a99}
\definecolor{lightgray}{HTML}{41555d}
\definecolor{lightblack}{HTML}{424c50}
\definecolor{orange}{HTML}{9b4400}
\definecolor{yellow}{HTML}{ffa631}
\definecolor{white}{HTML}{f0fcff}

\tcbuselibrary{skins, breakable, theorems}
\newtcolorbox{df}[2][]{
	colback=lightgray!5, colframe=lightblack,colbacktitle=lightblack, coltitle=white,title={#2},fonttitle=\bfseries,breakable
}

\newtcolorbox{thm}[2][]{
	colback=lightblue!5, colframe=darkblue,colbacktitle=darkblue, coltitle=white,title={#2},fonttitle=\bfseries,breakable
}

\newtcolorbox{eg}[2][]{
	colback=lightgreen!5, colframe=green,colbacktitle=green, coltitle=white,title={#2},fonttitle=\bfseries,breakable
}

\newtcolorbox{prf}[2][]{
	colback=orange!5, colframe=orange,colbacktitle=orange, coltitle=white,title={#2},fonttitle=\bfseries,breakable
}

\newtcolorbox{ext}[2][]{
	colback=yellow!5, colframe=yellow,colbacktitle=yellow, coltitle=white,title={#2},fonttitle=\bfseries,breakable
}

\newtcolorbox{rmk}[2][]{
	colback=lightred!5, colframe=red,colbacktitle=red, coltitle=white,title={#2},fonttitle=\bfseries,breakable
}

\lstset{language=Python}
\lstset{
	numbers=left,
	numberstyle=\tiny,
	keywordstyle=\color{green},
	commentstyle=\color{lightgray},
	frame=shadownbox,
	rulesepcolor=\color{lightblack},
	escapeinside='',
	xleftmargin=2em,aboveskip=1em,
	framexleftmargin=2em
	}

\title{Emory University\\\textbf{SOC 101 - Introduction to General Sociology \\Lecture Notes}}
\author{Jiuru Lyu}
\date{\today}

\rhead{\thepage}
\linespread{1.15}

\setcounter{secnumdepth}{4}
\setcounter{tocdepth}{4}

\begin{document}
\maketitle
\tableofcontents
\newpage

\section{The Sociological Perspective}
\subsection{What Is Sociology?}
\subsubsection{How Do We Define Sociology as Discipline?}
\begin{enumerate}
	\item Gladwell's "Food Fight":
	\begin{enumerate}
		\item Main points and arguments:
		\begin{itemize}
			\item Budget allocation $\rightarrow$ tradeoffs
			\item Food at Bowdoin College is a moral problem
			\item Financial aid vs. other aspects of spending
			\item Vassar $\rightarrow$ What's more important?
		\end{itemize} 
		\item Things You Didn't know before: Universities do not need to pay taxes.
		\begin{prf}{Why?}
			\begin{enumerate}
				\item Universities are non-profit organizations. \\Some universities buy lots of expensive properties to increase their endorsement, such as NYU. 
				\item Universities are providing a public good, so they ought to receive some bonus from society. However, they also need to pay back society with high-quality and fair education. $\rightarrow$ \textbf{\color{red}{Privilege leads to responsibilities.}}
			\end{enumerate}
		\end{prf} 
		\item This podcast is NOT sociology. Instead, it provides some sociological perspectives. $\rightarrow$ This podcast is not objection: 
		\begin{itemize}
			\item It adds a moral opinion.
			\item It lacks sufficient evidence: a) it only interviewed one student, and b) it only picks "food" as the topic.
		\end{itemize}
	\end{enumerate}
	\item Sociology as a discipline:
	\begin{enumerate}
		\item Definition of Sociology:
		\begin{df}{Sociology}
			\textbf{Sociology} is the systematic study and explanations of \\
			 - {social behavior}, \\
			 - {social groups}, and \\
			 - {society}.
		\end{df}
		\item Sociology is also interested in following contexts: 
		\begin{ext}{Sociology Definition - Extended}
			Sociology is also interested in how \textbf{social forces} influence or shape our \textbf{individual behavior}. 
		\end{ext}
	\end{enumerate}
	\item Sociology and other disciplines: Nature vs. Nurture debate
	\begin{enumerate}
		\item In Biology, we use Biology/genes to explain human behavior. 
		\item In Psychology, we use mental process to explain human behavior. It focuses on explanations WHIHIN individuals. 
		\item In Sociology, we use social interactions and environment to explain human behavior. It focuses on explanations in the EXTERNAL social realm. 
	\end{enumerate}
\end{enumerate}

\subsubsection{Connecting Society and the Individual: Fundamental Theorem of Sociology}
\begin{enumerate}
	\item Social context shapes human behavior. \\{\color{green}{Social context = Environment}}
	\item Individual behavior is deeply shaped (\textit{but not determined}) by social forces (social systems).
	\item Thinking "sociologically":
	\begin{enumerate}
		\item See beyond merely the individual.
		\item See behind assumptions of everyday life.
		\item Begin to think about the way society is structured. 
		\begin{eg}{Example of a Question for Thinking Sociologcially}
			"How are relationships organized? "\\
			"How are relationships between professors and students organized? "
		\end{eg}
	\end{enumerate}
	\item A summary of Sociology as a Discipline: 
	\begin{rmk}{Summary: Sociology as a discipline}
		\begin{itemize}
			\item Sociology recognizes the importance of individuals and the importance of biological and psychological factors.
			\item But, this is NOT ALL that is needed to explain human behavior.
			\item Social Context (Environment) is an inescapable factor in human life.
			\item Biology and Environment are inseparable.
		\end{itemize}
	\end{rmk}
\end{enumerate}

\subsubsection{Sociology is a Social Science}
\begin{enumerate}
	\item Social science or "common sense?"
	\begin{enumerate}
		\item Common sense:
		\begin{df}{Common Sense}
			Common science is sound  judgment not based on specialized knowledge. 
		\end{df}
		\item Science: 
		\begin{df}{Science}
			Science is a \textbf{\underline{system} of knowledge} covering the operation of general laws, obtained and tested through scientific method.
		\end{df}
		\item Scientific method: 
		\begin{df}{Scientific Method}
			Scientific method refers to \textbf{\underline{systematic}} pursuit of knowledge involving the recognition and formulation of a problem, the collection of \textbf{\underline{data}} through observation and experiment, and the formulation and testing of hypotheses.
		\end{df}
		\begin{ext}{Why do we need scientific methods? }
			We what to make sure that our study and research is replicable. 
		\end{ext}
		\item Sociology: 
		\begin{df}{Sociology - \textit{Refined Definition}}
			Sociology is the \textbf{\underline{systematic study}}, using the scientific method, to test hypotheses/questions about social institutions, social interactions, and social relationships.  
		\end{df}
		\begin{rmk}{Note on the Definition of Sociology}
			As indicated by the definition, sociology only focus on people as a group; sociology never studies individuals and individual behaviors. 
		\end{rmk}
	\end{enumerate}
	\item Anecdote V.S. Empirical Evidence: 
	\begin{enumerate}
		\item Anecdote: 
		\begin{df}{Anecdote}
			Anecdote is a usually short narrative of an interesting, amusing, or biographical incident. 
		\end{df}
		\item Evidence:
		\begin{df}{Evidence}
			Evidence is something that furnishes proof. 
		\end{df}
		\item Empirical: 
		\begin{df}{Empirical}
			The word "empirical" means "based on observation; capable of being verified or disproved by observation or experiment. "
		\end{df}
	\end{enumerate}
	\item Social science and generalizations: 
	\begin{enumerate}
		\item Category: 
		\begin{df}{Category}
			Category means to distinguish one thing from another; how we make distinctions between things. 
		\end{df}
		\item Generalization: 
		\begin{df}{Generalization}
			Generalization means the characteristic of objects within a category; defines similarities and differences with other categories. 
		\end{df}
		\item Stereotype: 
		\begin{df}{Stereotype}
			Stereotype is an exaggerated description applied to every person in some category. 
		\end{df}
		\item The purpose of social science is to develop accurate categorizations and generalizations about humans. 
		\item Sociology can predict: 
		\begin{enumerate}
			\item Which \underline{groups} will be \underline{more likely} to engage in certain types of behavior. 
			\item But, not which \underline{particular person} in the group will conduct certain types of behavior. - \textit{Sociology only studies humans as a group not human as an individual.}
			\item Sociology predicts categories (and generalizations about them), but not a precise individual's behavior.
		\end{enumerate}
	\end{enumerate}
\end{enumerate}


\subsection{How Do We Do Sociology? (Sociological Research Methods)}
\subsubsection{Sociology Research Model}
\begin{enumerate}
	\item Theory V.S. Research: 
	\begin{enumerate}
		\item Theory - Definition: 
		\begin{df}{Theory}
			\begin{enumerate}
				\item A systematic \textbf{explanation} for the observed facts that relate to a particular aspect of life. 
				\item General propositions about the relationship between two or more concepts (variable).
				$$A\  \xrightarrow[]{\text{ impact }}\ B$$
			\end{enumerate}
		\end{df}
		\item Empirical research: 
		\begin{enumerate}
			\item Definition: 
			\begin{df}{Empirical Research}
				Using \textbf{systematically gathered data} to prove a theory.
			\end{df}
			\item Sociologists conduct research - gather data about real people - to determine if theories are true depictions and predictions of reality. 
		\end{enumerate}
		\item Method: 
		\begin{df}{Method}
			The way we gather data for empirical research.\\
			{\color{green}\textit{Note: Your theory and your method must connect to each other. }}
		\end{df}
	\end{enumerate}
	\item The Sociology Research Model: 
	\begin{enumerate}
		\item Develop a Research Question: Select topic \& Define problem: 
		\begin{df}{Research Question (RQ)}
			The relationship between two or more concepts or variables. \\
			(How does A affect B? )
		\end{df}
		\item Theory: Review theoretical literature: 
		\begin{enumerate}
			\item What theories have been used to address this research question in the past?
			\item Choose specific theory (or theories) you want to test.
		\end{enumerate}
		\item Hypothesis: 
		\begin{df}{Hypothesis}
			A hypothesis is a statement of what you expect to find based on the theory. A hypothesis predicts a relationship between two or more concepts. 
		\end{df}
		\item Empirical test: The heart of a sociological research
		\begin{enumerate}
			\item Choose a Method (\textit{will be discussed later})
			\item Collect the data		
		\end{enumerate}
		\item Results: What did I find? 
		\begin{enumerate}
			\item Analyze results
			\item Present your findings - summarizing key points, and illustrating the results with:
			\begin{itemize}
				\item How do my findings tie into my hypotheses?
				\item How do my findings fit with previous research?
				\item What might be desirable further research?
			\end{itemize}
		\end{enumerate}
	\end{enumerate}
\end{enumerate}

\subsubsection{Different Social Research Method}
\paragraph{Survey Research}
\begin{enumerate}
	\item Survey - Definition
	\begin{df}{Survey}
		In survey research, sociologists collect data through asking people questions. They set questions, followed by a list of responses, in a survey. Most data from surveys are \textbf{quantitative} (numeric). 
	\end{df}
	\begin{eg}{Example of a Likert Scale}
		"One a scale of 1 to 5, ..."
	\end{eg}
	\item Importance things to consider when setting or conducting a survey research: 
	\begin{enumerate}
		\item Types of survey: 
		\begin{enumerate}
			\item Self-administered questionnaires (such as via email)
			\item Phone surveys
			\item In-person surveys (such as an interview)
		\end{enumerate}
		\item Population: 
		\begin{enumerate}
			\item Define the group you want to study
			\item Often dependent on your research questions and past literatures. 
		\end{enumerate}
		\item Sampling: 
		\begin{enumerate}
			\item How sample chosen is extremely important for results
			\item Sample should \textbf{represent} the entire Population
			\item Probability sampling: 
			\begin{df}{Probability Sampling}
				In simplest form, each person in the population has an equal chance of being chosen for the study. 
			\end{df}
		\end{enumerate}
		\item Question wording: 
		\begin{enumerate}
			\item Form, wording, and context of questions are important for accurate results.
			\item Complex issues need multiple questions. 
		\end{enumerate}
	\end{enumerate}
	\item Advantages and Disadvantages of Survey Research: 
	\begin{enumerate}
		\item Pros: 
		\begin{enumerate}
			\item Best way of learning about large populations.
			\item Data can be representative of the large population (with good sampling and response rate)
			\item Standardization of data
			\item Economy - cost
			\item Time
		\end{enumerate}
		\item Cons: 
		\begin{enumerate}
			\item Poorly worded questions yield poor data.
			\item Questions can be somewhat artificial (or superficial).
			\item Data can be potentially superficial.
			\item Difficult to gain a full sense of social processes in their natural settings. 
		\end{enumerate}
	\end{enumerate}
\end{enumerate}

\paragraph{Qualitative Research (Field Research)}
\begin{enumerate}
	\item Definition: 
	\begin{df}{Qualitative Research}
		Systematic, often long-term (around 1 to 10 years), observation of social phenomena in natural settings.
	\end{df}
	\item Most often, we use qualitative research for \underline{topics that are complex} and not easy to assess using surveys. 
	\item Data are qualitative - \textbf{Non-numerical}
	\begin{enumerate}
		\item Data collected in text form (not numerical scales)
		\item The text from notes and interviews are \textbf{coded} and organized around themes.
		\item Then they are systematically analyzed for patterns
		\item Summaries are usually non-numerical (although one may count how many people mentioned particular themes).
	\end{enumerate}
	\item Observation as a type of Qualitative Research: 
	\begin{enumerate}
		\item Definition: 
		\begin{df}{Observation}
			Observe a group of people or a certain event.
		\end{df}
		\item Data are collected through: 
		\begin{enumerate}
			\item Field journal
			\item interviews
			\begin{ext}{Different Types of Questions Asked}
				Interviews use open-ended questions, whereas surveys ask close-ended questions.
			\end{ext}
		\end{enumerate}
		\item Types of Observation: 
		\begin{enumerate}
			\item Direct observation: 
			\begin{enumerate}
				\item Researcher observes a social group as an \textbf{outsider}.
				\item Does not become part of the group in any way.
				\item Usually no influence on group dynamics.
			\end{enumerate}
			\item Participant Observation
			\begin{enumerate}
				\item Researcher goes beyond mere observation to participate in the group they are studying. 
				\item Potential for influencing the group.
			\end{enumerate}
		\end{enumerate}
	\end{enumerate}
	\item In-Depth Interviews as a type of Qualitative Research: 
	\begin{enumerate}
		\item Usually a part of observation studies. 
		\item Can also be done as a separate method.
		\item Explore a topic at a more complex level. 
		\item Use when you want \textbf{depth} and \textbf{diversity} of views. 
		\item Interviewer has very general questions and has the respondent talk at length. 
		\item Potential for interviewer bias. 
	\end{enumerate}
	\item Advantages of Qualitative Research: 
	\begin{enumerate}
		\item Able to study nuances of attitudes and behaviors.
		\item Able to study whole group - defining social structure of group.
		\item More \textbf{depth} and better understanding of what is studied.
		\item Can be inexpensive. 
	\end{enumerate}
	\item Disadvantages of Qualitative Research: 
	\begin{enumerate}
		\item Time-intensive
		\item Potential for subjectivity and bias.
		\item Conclusions are regarded as \textbf{suggestive} rather than \textbf{definitive}.
		\item Smaller sample - \textit{less generalizability} to larger populations.
	\end{enumerate}
\end{enumerate}

\paragraph{Experiment}
\begin{enumerate}
	\item This is the most rigorous controllable of the methods. 
	\item You define a situation you want to test and then create that situation in a lab. 
	\begin{df}{Groups in Experiments}
		\textbf{Experimental group}: exposed to test factor.\\
		\textbf{Control group}: identical in terms of types of individuals, but not exposed to test factor. 
	\end{df}
	\item Advantages of Experiments: 
	\begin{enumerate}
		\item Control over variables. 
		\item Can test specific influences over a specific situation.
	\end{enumerate}
	\item Disadvantages of Experiments: 
	\begin{enumerate}
		\item Artificial environment: how do you know lab settings will be the same as in real life? 
		\item Limited in scope
		\item Ethical concerns - Moral concerns of human subject experiments \textit{(This has been addressed, by and large, through Institutional Review Boards (IRBs))}
	\end{enumerate}
	\begin{eg}{Milgram's Experiment}
		\begin{enumerate}
			\item Introduction
			\begin{enumerate}
				\item Psychological experiment in early 1960s. 
				\item Conducted variations with different groups/samples. 
				\item Slightly different from most experiments today - no control group. 
			\end{enumerate}
			\item Purpose: 
			\begin{enumerate}
				\item Involved observing people's willingness to harm others when following orders. 
				\item Wanted to see threshold for when people would take personal responsibility and disobey superior. 
			\end{enumerate}
			\item Why do people obey authority figures? 
			\begin{enumerate}
				\item Assume their authority figures. 
				\item Fear reprimand/anticipate reward.
				\item Pass responsibility on to someone else ("just following orders")
				\item etc. (look at social context for answers...)
			\end{enumerate}
			\item Effect on participants: 
			\begin{enumerate}
				\item Through a later survey, 83\% said they were glad they had participated. 
				\item Yet, many personally experienced problems during the experiment. 
				\item Some felt shame; others justified their behavior. 
			\end{enumerate}
			\item Critiques of method: 
			\begin{enumerate}
				\item Its psychological effects on the participants. 
				\item Deception of the participants. 
			\end{enumerate}
		\end{enumerate}
	\end{eg}
\end{enumerate}

\subsubsection{Ethics in Research}
\begin{enumerate}
	\item These studies like Milgram's experiment has led to 
	\begin{enumerate}
		\item Codes of ethics to protect human subjects
		\item Institutional Review Boards
	\end{enumerate}
	\item A note on peer review
\end{enumerate}

\subsubsection{Causation vs. Correlation}
\begin{enumerate}
	\item Definitions: 
	\begin{df}{Causation and Correlation}
		\textbf{Correlation}: A relationship between two variables.\\
		\textbf{Causation}: One variable causes another. 
	\end{df}
	\item Element of time - which came first
	\item Intervening - or third - variable
\end{enumerate}
\begin{eg}{Case Study: \textit{Bad Feminist} by Roxane Gay}
	\begin{enumerate}
		\item Why read this article (as it is not sociological research)? 
		\begin{enumerate}
			\item Gay emphasizes the importance of cultural awareness and developing cross-cultural knowledge and skills. 
			\item She wants us to pay attention - critically and carefully - to the world around us. 
			\item She also articulates the importance of grace - wherein people are allowed to make mistakes, learn, and evolve. 
			\item It is an educational approach to life. 
			\item it is not an approach wherein we imagine we know everything already and attack other who "don't get it."
			\item To a certain degree, she is also exploring "imperfection."
		\end{enumerate}
		\item What is this essay about? 
		\begin{enumerate}
			\item Gay confronts the reductive nature of feminism and the stereotypes it produces. 
			\item She also addresses her own reservations towards embracing feminism itself. 
			\item One thing not in the essay (but she does take up elsewhere) is that \underline{feminism is also a social movement}. 
		\end{enumerate}
		\item Exercise: 
		\begin{enumerate}
			\item She points to Judith Butler to articulate the ways in which we all "perform" our gender. 
			\item Judith Butler: American Philosopher and gender theorist.
			\item Is there a "right" way to be a woman? Or a "right" way to be a man? 
		\end{enumerate}
		\item What is sociological about feminism? 
		\begin{enumerate}
			\item Sociology often explores the unequal distribution of power and resources, feminist sociology studies power in its relation to gender. 
			\item Feminist scholars study a range of topics, including sexual orientation, race, economic status, and nationality. 
			\item At the core of feminist sociology is the idea that, in most societies, women have been systematically oppressed and that \textbf{men have been historically dominant}. (This is referred to as \textbf{patriarchy}.)
			\item All of these categories - including 'feminist' - are being controlled by stereotypes and judgments. 
		\end{enumerate}
		\item Gay on Categories: 
		\begin{enumerate}
			\item "We are categorized and labeled from the moment we come into this world by gender, race, size, hair color, eye color, and so forth."
			\item "Again, we see this fear of categorization, this fear of being forced into a box that cannot quite accommodate a woman properly." [and, a man ... and other gender categories ...]
		\end{enumerate}
		\item \textit{\textbf{Gay's critique of many white feminists }}
		\begin{enumerate}
			\item Not interested in the issues unique to women of color - having to work against a different set of stereotypes ("angry black woman', etc.); are oftentimes dismissive. 
			\item Worried that a recognition of differences among feminists will lead to divisiveness.
			\item Their argument that black women need to do the work of making feminist organizations more inclusive.
			\item Appropriation of material (ideas, etc.)
			\item Gay notes that there are problems with feminism - it is not perfect: There are problems with all social movements, and as they expand, they are often characterized by divisiveness. 
		\end{enumerate}
	\end{enumerate}
\end{eg}

\subsection{Basic Sociological Concepts - Social Structure: Statuses, Roles, and Norms}
\subsubsection{How Do We Understand Social Context?}
\begin{enumerate}
	\item Social order: 
	\begin{enumerate}
		\item Why does social context matter? Because it provides \textbf{order} in our world. 
		\item Sociologists often study the processes that enable social order; the formal and informal rules that allow society to function.
		\item Social order is created and maintained through: 
		\begin{enumerate}
			\item Laws and formal rules
			\item Informal social processes/factors
		\end{enumerate}
	\end{enumerate}
	\item Social context defined: 
	\begin{enumerate}
		\item Definition: 
		\begin{df}{Social Context (Social Environment)}
			External reality formed by interactions between individuals.
		\end{df}
		\item Notes on the definition: 
		\begin{enumerate}
			\item Can range from \textbf{micro (family)} to \textbf{macro (nation)}. 
			\item Individuals live in multiple contexts at the same time (e.g., family, religious group, city, nation). 
			\item Individuals experience the \textbf{primacy} of one context over others at particular times. 
		\end{enumerate}
		\item Sociologists seek to uncover, understand, and explain different social contexts, as well as their influence over individual behavior. 
	\end{enumerate}
	\item Social Structure and Culture: 
	\begin{enumerate}
		\item They way sociologists study social context is by analyzing two major aspects of it: 
		\begin{enumerate}
			\item Social structure
			\item Culture
		\end{enumerate}
		\item Social Structure: 
		\begin{df}{Social Structure}
			The predictable rules or patterns of interaction between people and groups.
		\end{df}
		\begin{enumerate}
			\item How relations among people are structured.
			\item How parts of society are related. 
		\end{enumerate}
		\item Culture: 
		\begin{df}{Culture}
			What the structures/interactions mean.
		\end{df}
		\begin{enumerate}
			\item Shared system of meaning that exists in any society/social context.
			\item Shared way of life - shared way of doing things/practices that undergird structure. 
		\end{enumerate}
		\item Social Structure and culture enable social order $\rightarrow$ allows us to know how to behave. 
		\item Social Structure and culture co-exist in social contexts. 
	\end{enumerate}
\end{enumerate}

\subsubsection{Social Structure: Status and Role}
\begin{enumerate}
	\item Status
	\begin{enumerate}
		\item Definition: 
		\begin{df}{Status}
			Social \textbf{positions} people occupy. \\
			Any social position that has rights, obligations, and expectations that go along with that position. 
		\end{df}
		\item Ascribed status: 
		\begin{df}{Ascribed Status}
			A position \textbf{given} at birth or assigned at different stages of life. 
		\end{df}
		\item Achieved status: 
		\begin{df}{Achieved Status}
			A position \textbf{acquired} through personal effort. 
		\end{df}
		\item Status-set: combination of various statuses. It is the sum of positions that we occupy in society. 
		\item Master status: when one status assumes a certain priority and appears to override other statuses that you hold. 
	\end{enumerate}
	\item Role
	\begin{enumerate}
		\item Definition: 
		\begin{df}{Role}
			The \textbf{expressions and behavior} of a person who occupies a particular status. 
		\end{df}
		\item Roles involve expressions and behaviors that are appropriate to the status. 
		\item Roles define our interactions with occupants of other statuses. 
		\item Roles involve not just actions, but also expressions of feeling and emotion. 
		\item Statuses vs. Roles
	\end{enumerate}
	\item Status and Role in Interaction: Goffman
	\begin{enumerate}
		\item Goffman: People in their everyday interactions are like \textbf{actors performing on a stage}
		\item Social interaction = theatrical performance\\
		\textbf{Status} = Character in a play\\
		\textbf{Role} = Script; the dialogue and action of the character. 
		\begin{eg}{Our "performance" includes}
			\begin{enumerate}
				\item The way we dress (costume)
				\item The objects we carry (props)
				\item Our tone of voice and gestures (manner)
				\item Performances vary according to where we are (context/set)
			\end{enumerate}
		\end{eg}
		\item \textbf{"\textit{Presentation of Self}" = each individuals' performance.}
		\begin{enumerate}
			\item Individuals can influence the performance.
			\item As we present ourselves in everyday situations, we reveal information to others.
			\item We try to create specific impressions about ourselves. 
			\item This is also called "\textbf{impression management}"
			\item It has several distinct elements: 
			\begin{enumerate}
				\item \textbf{Defining} the situation (set) a certain way
				\item \textbf{Presenting} a certain \textbf{status}
				\item \textbf{Managing} how we \textbf{play our role}
				\item \textbf{Working consensus} - overall agreement on definition of the situation. 
			\end{enumerate}
		\end{enumerate}
		\item Stage 1 - "Definition of the Situation":  This is like the stage/set
		\begin{enumerate}
			\item We know something about the play simply by seeing the set or setting. 
			\item We "define" interactions with other people based on information we initially perceive about the setting: 
			\begin{enumerate}
				\item Physical surroundings
				\item Props
			\end{enumerate}
			\item We also "define" the situation based on people's \textbf{statuses}.
			\item When we interact, we are constantly searching for \textbf{cues to their statuses}. 
		\end{enumerate}
		\item Stage 2 - "Presentation of Self": Here is where the play begins: the action or interaction begins
		\begin{enumerate}
			\item \textbf{We play the role connected to our status} (or the status we would like to have):
			\begin{enumerate}
				\item The techniques that people use to get others to see them in a certain light. 
				\item How we give information to others that we interact with - express ourselves.
				\item Expressiveness of the individual: \\
					Expressions \textbf{given} = verbal communication\\
					Expressions \textbf{given off} = non-verbal communication - gestures; facial expressions
			\end{enumerate}
		\end{enumerate}
		\item Stage 3 - "Managing the Situation": Here, individuals can "manage" aspects of the performance in order to get people to see them a certain way - Going beyond the mere script (role)
		\begin{enumerate}
			\item Individual who is presenting self can manage the situation to some extent
			\item To control conduct of others, especially their treatment of you. $\rightarrow$ \textbf{Must get others to define the situation the way you do}. 
			\item Can do this by managing your expressions.
		\end{enumerate}
		\item Stage 4 - "Working Consensus": 
		\begin{enumerate}
			\item For interactions to be successful, people must agree about the definition of the situation. 
			\item Must agree on statuses and roles for particular context. 
			\item Definition: 
			\begin{df}{Working Consensus}
				Together the participants contribute to a single overall-definition of the situation, which involves: 
				\begin{enumerate}
					\item Not so much real agreement as to what exists.
					\item But rather real agreement as to whose "definition" will be honored in that situation. 
				\end{enumerate}
			\end{df}
		\end{enumerate}
		\item Summary of Goffman: 
		\begin{rmk}{Summary of Goffman - Interaction is like play}
			We can present ourselves in a certain way, manage impressions of ourselves, and define situation:
			\begin{enumerate}
				\item Because of our shared understanding of statuses. 
				\item Because those statuses have roles and expectations attached to them.
				\item And expectations associated with roles are norms. 
			\end{enumerate}
		\end{rmk}
	\end{enumerate}
\end{enumerate}


\subsubsection{Social Structure: Norms and Social Control}
\begin{enumerate}
	\item Norms: 
	\begin{enumerate}
		\item Definition: 
		\begin{df}{Norms}
			The shared \textbf{rules and expectations} that govern our behavior.
		\end{df}
		\begin{enumerate}
			\item Norms vary depending on context or situation.
			\item Norms are most often tied to \textbf{statuses}, and govern \textbf{roles}.
			$$\text{Status}\ \longrightarrow\ \text{Role}\ \longrightarrow\ \text{Norms}$$
		\end{enumerate}
		\item Norms inform us how we are to act, toward whom, where, and when. 
		\item And, not only how we are to act, but to anticipate in others. 
	\end{enumerate}
	\item Social Control: 
	\begin{enumerate}
		\item Norms make our interactions orderly and predictable. 
		\item Thus, society has mechanisms to ensure that people conform to norms. 
		\item Definition: 
		\begin{df}{Social Control}
			The ways that society (people or institutions) attempts to keep people in line with social norms. 
		\end{df}
		\item Types of social control: 
		\begin{df}{Positive sanction}
			Rewards for approved behavior. 
		\end{df}
		\begin{df}{Negative sanction}
			Punishment for disapproved behaviors.
		\end{df}
		\item Mechanisms of social control: 
		\begin{enumerate}
			\item Vary in severity
			\item Vary informality
		\end{enumerate}
	\end{enumerate}
	\item Further characteristics of norms: 
	\begin{enumerate}
		\item Norms are comprehensive
		\begin{enumerate}
			\item Norms shape all our behavior.
			\item Even our perceptions - how we perceive things have predictable patterns. 
		\end{enumerate}
		\item Norms vary in intensity: Norms vary in the intensity of their moral significance.  
		\begin{enumerate}
			\item "Mores": Norms that have great moral significance; very important to a society. 
			\item "Folkways": Norms for routine or casual interaction in a society. 
		\end{enumerate}
		\item Limits to Norms: 
		\begin{enumerate}
			\item Norms can be manipulated: 
			\begin{enumerate}
				\item As Goffman shows, we can regulate our "presentation of self" in society. 
				\item We can also engage in "role distance"
				\item Show we aren't really occupier of a particular status. 
			\end{enumerate}
			\item Norms are not always clearly defined. 
			\begin{enumerate}
				\item Case of unclear script
				\item Very often with new statuses in society
			\end{enumerate}
			\item Status conflict: \\
			In some cases, roles are incompatible because statuses are in conflit. 
	\end{enumerate}
	\item Norms in Interaction: Goffman
	\begin{eg}{Example: Cahill's article on "Front Stage/Back Stage" of bathrooms}
	\begin{enumerate}
		\item Front stage is public place where we expect to "perform".
		\item Back stage is area where we get ready for our front stage performances. 
		\item Different "norms" govern these spaces. 
		\item Questions to consider: 
		\begin{enumerate}
			\item What is the purpose of the public bathroom?
			\item What are the two performance regions and how are they regarded? 
			\item How do people interact in public bathrooms - what are the bathroom "interpersonal rituals"?
			\item What are the norms that govern interaction in this particular social context, with its distinct performance regions? 
			\item Why do we have these norms? 
			\item What is a secondary purpose of the public bathroom? 
		\end{enumerate}
		\item In summary, why do we do certain things in a bathroom? \\
		The norms in this private yet public setting: 
			\begin{enumerate}
				\item Show that we are loyal to the "behavioral guidelines" of our society (norms/values about privacy and bodily functions)
				\item Contribute to social order
			\end{enumerate}
		\end{enumerate}
		\end{eg}
	\end{enumerate}
\end{enumerate}

\subsection{Basic Sociological Concepts - Culture}
\subsubsection{Understanding Culture}
\begin{enumerate}
	\item Working definitions: 
	\begin{df}{Culture}
		\textbf{Shared system of meaning}: what social structures/interactions mean. 
	\end{df}
	\begin{enumerate}
		\item Shared system of meaning that exists in any society; undergirds social structure. 
		\item The way we make distinctions between good and bad; important or not. 
		\item Classifications; distinctions; values. 
	\end{enumerate}
	\item Links between culture and social structure. 
	\begin{enumerate}
		\item Social structure are culture that are intertwined, but they focus on different aspects of a social life: 
		$$\text{(Social Structure)}\qquad\qquad\qquad\text{Statuses}\ \longrightarrow\ \text{(Roles)}\ \longrightarrow\ \text{Norms}$$
		$$\text{(Culture)}\qquad\qquad\qquad\qquad\text{prestige}\ \longleftrightarrow\ \text{Values} $$
		\item Values: Values often accompany \textbf{norms}
		\begin{enumerate}
			\item They justify norms and \textbf{provide believable reasons as to why we should conform}.
			\item Values define what is worthwhile or important. 
			\item They are shared believes of what is good or normal. 
		\end{enumerate}
		\item Prestige: Prestige often accompanies \textbf{status}\\
		Based on values, prestige concerns \textbf{specific activities} or statuses that \textbf{the group defines as important and good}. 
	\end{enumerate}
	\item Different levels of culture: 
	\begin{enumerate}
		\item Explicit: 
		\begin{enumerate}
			\item Culture operates in an explicit fashion.
			\item Areas that explicitly/obviously deal with meaning.
			\item Material culture
			\item Values/Beliefs: 
			\begin{itemize}
				\item Beliefs that people share regarding what is good, beautiful. 
				\item Values define what is worthwhile, important. 
			\end{itemize}
		\end{enumerate}
		\item Implicit: 
		\begin{enumerate}
			\item It also operates in an implicit fashion. 
			\item Parts of life that have meaning, but that we "take-or-granted"
			\item Sometimes, we hold values and beliefs that are implicit: we do NOT recognize them as values/cultures. 
		\end{enumerate}
	\end{enumerate}
\end{enumerate}

\subsubsection{Culture shock and Ethnocentrism}
\begin{enumerate}
	\item Ethnocentrism
	\begin{df}{Ethnocentrism}
		Tendency to regard our way of life as the \textbf{right} way. 
	\end{df}
	\item Cultural relativity: 
	\begin{enumerate}
		\item Stance taken by social scientists (opposite of ethnocentrism)
		\item Different societies create different values and different systems of meaning. 
		\item Difference does not equal "right" or "wrong"
	\end{enumerate}
	\item Culture Shock: 
	\begin{df}{Culture Shock}
		A feeling of confusion, doubt, or nervousness (or disorientation) caused by being in a place (such as a foreign country) that is different from what people are used to.
	\end{df}
	\item Example: Nacirema
\end{enumerate}
 
\subsubsection{Social Construction of Reality}
Our world is "socially constructed"
\begin{enumerate}
	\item "Construction" aspect: 
	\begin{enumerate}
		\item Nothing contains meaning in-and-of-itself
		\item Humans "construct" or create meaning; including categories/distinctions that are important (e.g. gender/race)
	\end{enumerate}	
	\item "Social aspect: 
	\begin{enumerate}
		\item Humans create meaning together. 
		\item Meaning is created through social interaction/social processes.
		\item People decide together on meanings to assign categories/distinctions, events, objects. 
	\end{enumerate}
	\item Example: How to Become a Batman Podcast
	\begin{eg}{ }
		How is blindness a "social construction"?
		\begin{enumerate}
			\item our expectations of blind people from one another (The social world)
			\item What does the meaning of blindness? Can blind people "see"? Or does a blind people need help from non-blind people?
			\item The meaning we give to "blindness" effects both the blind and the non-blind. It impacts our interactions. 
			\item In fact, the non-blind constrain the behaviors of the blind as a consequence. 
			\item We change our expectations of human beings because of meanings we "take-for-granted".
			\item Meaning - what something means - can change our behavior.  
		\end{enumerate}
		Questions to think about: 
		\begin{enumerate}
			\item Where did your thoughts/beliefs about blind people come from?
			\item How do you, personally, respond when you see someone who might be blind?  Why do you respond the way you do?
			\item Daniel and Adam got “lumped together” in their school.  People would mix them up, thinking one was the other.  
			\item Has this ever happened to you?  Why were you “lumped together” with someone else? How did you respond?  Why do you think you responded the way you did?
			\item Do you live in a world that believes you can’t do certain things? (Think of the man who worked in a paint factory)
			\item What would happen if we changed our expectations of what blind people can do?  Do you think we should?
			\item Have you ever been the subject of someone else’s lower expectations of you?  How did you respond?
		\end{enumerate}
	\end{eg}
	\item Example: The 7 Day Week
	\begin{eg}{ }
		\begin{enumerate}
			\item Reality of the 7-Day Week: 
			\begin{enumerate}
				\item The week is ubiquitous, but taken-for-granted 	$\rightarrow$ Implicit culture
				\item The week gives us a "temporal map" to organize our lives
			\end{enumerate}
			\item The week is a social construction of reality: 
			\begin{enumerate}
				\item It does not correspond to any naturally occurring phenomenon.
				\item It is made by people, together (part of culture): 
				\begin{enumerate}
					\item Origins of the 7-day week: 
					\begin{itemize}
						\item Judaism
						\item Astrology	
					\end{itemize}
					\item Attempts to change the 7-day week: Soviet Union\\
					These attempts did not work: Because culture was so embedded in people's lives that it seemed "inevitable."
				\end{enumerate}
			\end{enumerate}
		\end{enumerate}
	\end{eg}
	\item Social Construction of Reality
	\begin{enumerate}
		\item Our world is "socially constructed."
		\item People create meaning together. 
		\item Because meaning are created by people, \textbf{people can change them}. (This usually only happens in extraordinary circumstances.)
	\end{enumerate}
	\item Meaning, Behavior, and Norms
	\begin{enumerate}
		\item New meanings (new social construction of reality) will produce new behaviors.
		\item All groups must have norms or rules that govern these new behaviors.
		\item Without norms or rules (without structure), people do not know how to relate to each other, and we have chaos. 	
	\end{enumerate}
	\item Example: Henslin, Survivors of F-227
	\begin{eg}{How is the "social construction of reality" illustrated with the crash survivors in the Andes? }
		\begin{enumerate}
			\item Social Construction of Reality
			\begin{itemize}
				\item What \underline{deeply held \textbf{cultural meaning}} is at the heart of this situation? 
				\item How do their normal circumstances change - \underline{what is their \textbf{new social context}}? \\
					Why does this new social context challenge their deeply held values? 
				\item How does \underline{cultural meaning change because of new social context}?
			\end{itemize}
			\item Meaning, Behavior, and Norms
			\begin{itemize}
				\item After meaning changes, \underline{how does group behavior change}?
				\item \underline{How do norms develop} to guide this new behavior? 
				\item What norms did the group put into place? 
			\end{itemize}
		\end{enumerate}	
	\end{eg}

\end{enumerate}


\subsection{The Case of Love: Culture, Social Structures, and Sentiment}
We look at how social factors shape two aspects of romantic love in the US: Love and Marriage \& Sex without Love
\subsubsection{Who do you love?}
\begin{enumerate}
	\item Isn't it very individual and personal? 
	\begin{enumerate}
		\item All societies have \textbf{distinct social patterns} associated with love.
		\item Our practices and beliefs about love are \textbf{socially constructed}.
	\end{enumerate}
	\item The Social Construction of Love
	\begin{enumerate}
		\item Eva Illouz's \textit{Consuming the Romantic Utopia}
		\begin{enumerate}
			\item Helps illustrate the \textbf{social construction} of courtship/dating.
			\item The changing space of courtship.
			\begin{itemize}
				\item Moved from the private sphere to the public sphere
				\item Dating involves negotiating public spaces; knowing how to conduct oneself: Manners, culturally constrained behaviors
				\begin{eg}
					Country club, proper attire, adhering to the social norms of a particular group or subculture. 
				\end{eg}
				\item The economization of social relationships: now \textbf{consumption} played a huge role in dating: How much money do we spend? Where do we spend it? 
			\end{itemize}
		\end{enumerate}
		\item Viviana Zelizer's \textit{The Purchase of Intimacy}
		\begin{enumerate}
			\item Love is also a system that has its equivalence in goods, services, and entitlements. 
			\begin{eg}{How do you know when you're on a date?}
				\begin{enumerate}
					\item What was the meaning of the consumption we now saw in dating? 
					\item What constitutes a gift and what constitutes payment? How do you know? 
				\end{enumerate}
			\end{eg}
			\item We have cultural templates that allow us to understand: 
			\begin{itemize}
				\item the intimacy level between two people
				\item the behavioral boundaries of a relationship
				\item the economic boundaries of a relationship between two people 
			\end{itemize}
			\item These boundaries are regulated by institutions, emotions, time, and the law. 
		\end{enumerate}
	\end{enumerate}
	\item Social Factors and Mate Selection: Choice of mate is linked to social structure
	\begin{enumerate}
		\item Patterns of mate selection: 
		\begin{enumerate}
			\item Definitions: 
			\begin{df}{Endogamy}
				Endogamy refers to marriage within a social group	
			\end{df}
			\begin{df}{Exogamy}
				Exogamy refers to marriage outside a social group	
			\end{df}
			\item Definition: 
			\begin{df}{Social Stratification}
				A group's system of ranking.\\ 
				\textit{Note: Society's ranking of different statuses (large groups of people) according to a hierarchy, often based on prestige/values.}
			\end{df}
		\end{enumerate} 
		\item The US patterns: 
		\begin{enumerate}
			\item Social class: Most people marry within their own social class.
			\item Education: People marry others with similar education.
			\item Age: Usually, 2-3 years difference. Societies also define the right age to get married. 
			\item Religion: Most marry within the same religious group, but less endogamous than other categories. 
			\item Race: This is one of the most endogamous categories, but the proportion varies by race. 
		\end{enumerate}
		\item Explanations for endogamous patterns: 
		\begin{enumerate}
			\item Preferences play a role: 
			\begin{itemize}
				\item Individuals have preference for certain characteristics in a spouse.
				\item Our preferences are shaped by our social groups and statuses
			\end{itemize}
			\item Indirect regulation of mate selection
			\begin{itemize}
				\item "Formally Free" mate selection, but \textbf{indirect regulation} through \textbf{restriction of social interaction}
				\begin{enumerate}
					\item Influence of third parties (parents)
					\item Propinquity (proximity: e.g., residential segregation; school; workplace)
					\item Sanctions (e.g., religious)
				\end{enumerate}
				\item Those indirect regulations are ways to assure you "marry the right person."
			\end{itemize}
			\item Regulation of mate selection ensures social control 
			\begin{itemize}
				\item Without controls over mate selection - would have social chaos
				\item Different interpretations of social control over love. 
				\begin{eg}{William Goode's \textit{The Theoretical Importance of Love}}
					\begin{enumerate}
						\item Free love threatens class system
						\item Marriage between different class groups would disrupt social status of group
						\item Mate selection is most important for those with power and property. So, they place more restrictions on children and their social interactions.
						\item These mate selections patters serve to reinforce ad reproduce your \textbf{position in social structure}.
					\end{enumerate}
				\end{eg}
			\end{itemize}
		\end{enumerate}
	\end{enumerate}
\end{enumerate}

\subsubsection{What is Love? Definitions and meaning of Love}
\begin{enumerate}
	\item Love and Marriage
	\begin{eg}{Swidler: Talk of Love \textit{(The link between love and marriage and how we use culture to socially construct those concepts)}}
		\begin{enumerate}
			\item Background of book (Introduction)
			\begin{enumerate}
				\item Interviewed 88 middle-class men and women from suburban areas San Jose, CA.
				\item Not typical, but "Proto-typical" Americans
				\item "Middle-class" culture tends to be the dominant culture in our country - or "mainstream" culture
				\item Most of the participants were in their 30's and 40's and were or had been married. 
			\end{enumerate}	
			\item Culture nations of love: what is the cultural meaning of "romantic love" in the US? Two different views of love emerge in respondents' interviews. 
			\begin{enumerate}
				\item Love Myths - Mythic Love
				\begin{enumerate}
					\item Historical origins: Courtly Love Tradition
					\item Bourgeois tradition reshapes courtly love
					\begin{itemize}
						\item A decisive choice (love at first sight)
						\item A unique other (one true love)
						\item Overcoming obstacles (marrying for love, not money)
						\item Love lasts forever ("happily ever after")	
					\end{itemize}
				\end{enumerate}
				\item Real Love - Prosaic Realism
				\begin{enumerate}
					\item Participants often de-bunk the notion of mythic love.
					\item Offer an alternative cultural view of "real love": 
					\begin{itemize}
						\item Love grows slowly; is often ambivalent and confused
						\item One can love many people in a variety of ways
						\item Love should be based on compatibility and practical traits that make good partners.
						\item Love does not necessarily last forever. 
					\end{itemize}
					\item Love involves emotional sharing, communication, often equality, respect.
					\item Working at the relationship day-to-day
				\end{enumerate}
				\item People talk about both these views of love when talking about what love means to them
			\end{enumerate}
		\end{enumerate}
	\end{eg}
	\item Sex without Love
	\begin{enumerate}
		\item US in the 19th and 20th century \\
		Sex = Love (form Judeo-Christian culture)
		\begin{enumerate}
			\item Norm: Sex is part of committed loving relationship
			\item Values: Sex is sacred part of love relationship between monogamous couple
			\item So, sex without love is deviant
			\item In 1950's, Kinsey survey showed that more people engaged in casual sex than was expected.
			\item More recent patterns of sexual behavior
			\item Still, the US norm is against sex without loving commitment. 
		\end{enumerate}
		\item 21st century
		\begin{eg}{Case Study: Lambert et. al. \textit{Pluralistic Ignorance and Hooking UP}}
			\begin{enumerate}
				\item Background
				\begin{itemize}
					\item In the post, research has assumed that sex without love is a problem (goes against norms). 
					\item Today, on college campuses, this kind of behavior ("hooking up") has become normative.
					\begin{df}{Hooking up}
						Hooking up occurs when two people who are casual acquaintances engage in some forms of sexual behavior with the expectation of no future commitment. 	
					\end{df}
				\end{itemize}
				\item Background studies on "hooking up"
				\begin{itemize}
					\item Prevalence: What is the prevalence of hooking up? 
					\item Bad experiences: What "bad experiences" do students describe? Gender differences in bad experiences? 	
				\end{itemize}
				\item Theory: Pluralistic Ignorance
				\begin{itemize}
					\item Perceptions of other's attitudes: Group members believe that \textbf{others} in their group (especially leaders or popular people) \textbf{endorse a particular norm}). 
					\item Own attitudes: yet, they believe their \textbf{own personal attitudes} are different from the norm.
					\item Actual behavior: however, they go along with the norm because: 
					\begin{itemize}
						\item Desire to fit in with the group.
						\item Each person thinks that they are \textbf{the only one who has conflict} between their \textbf{personal attitudes} and their \textbf{actual behavior}
					\end{itemize}
				\end{itemize}
				\item Methods: 
				\begin{itemize}
					\item Survey
					\item Sample of 175 female and 152 male undergrads at mid-sized southeastern public university.
					\item Convenience sample (library; residence halls)
				\end{itemize}
				\item Results: 
				\begin{itemize}
					\item 77.7\% of women said they had hooked up
					\item 84.2\% of men said they had hooked up
					\item The results with regard to comfort level with hooking up? 
					\item Same-sex peers: How did men and women rate their comfort levels (self-ratings) as compared to others of same sex (peer-ratings)?
					\item Opposite-sex peers: How did men and women rate the comfort levels of the opposite sex? Did they over- or under-estimate comfort levels? 
					\item Gender differences: What were the overall gender differences? 
				\end{itemize}
				\item Conclusions: 
				\begin{itemize}
					\item Findings support the theory of pluralistic ignorance:
					\begin{itemize}
						\item Hooking up has become a norm on college campuses
						\item Most students think other people are comfortable with it - more comfortable than they are themselves. 
					\end{itemize}
					\item Potential consequences of gender differences: 
					\begin{itemize}
						\item Potential for sexual assault of women
						\item Why? 
					\end{itemize}
				\end{itemize}
				\item Critiques of the study? 
			\end{enumerate}	
		\end{eg}

	\end{enumerate}	
\end{enumerate}





\subsection{The Case of Death: Culture, Social Structure, and Fear}

\section{Individuals and Social Interaction}
\subsection{Socialization: Development of the Self}
\subsection{Deviance}

\section{Groups and Society}
\subsection{Social Class}
\subsection{Race and Ethnicity}
\subsection{Organizations}
\subsection{The sociology of Work}

\end{document}
	
	
